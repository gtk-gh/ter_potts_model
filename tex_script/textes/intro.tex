\addcontentsline{toc}{chapter}{Introduction}
\chapter*{Introduction}

\begin{document}
Etant étudiants en master Mathématiques et Application dans la section Ingénierie Statistique Numérique, l'étude d'évolution de cluster et de simulation grâce à des algorithmes MCMC\footnote{Markov chain Monte Carlo} nous semble assez logique.
Le but de notre TER était de réaliser une classification non supervisée d'un ensemble de point en utilisant le modèle de Potts.
On a donc procédé en deux étapes une premiere étape qui était l'implémentation d'un programme informatique en C++ et en Python.
Ces deux langages sont très utiles pour la modélisation qui se fera en Python et pour l'algorithmique qui se fera en C++.
Puis une deuxième partie qui est l'étude de l'algorithme de Metropolis-Hasting, sa convergence et l'utilité de cet algorithme dans un tel modèle.

\end{document}