\documentclass[12pt,a4paper]{report}
\usepackage[utf8]{inputenc}    %mettre utf8 ? la place de latin si vous utilisez utf8
\usepackage[T1]{fontenc}
\usepackage{setspace}
\renewcommand{\baselinestretch}{2}
\usepackage[francais]{babel}
\usepackage[maxlevel=3]{csquotes}
\usepackage[backend=bibtex,style=verbose-trad1,isbn=false]{biblatex}
\DefineBibliographyStrings{french}{in={dans},inseries={dans}}
\usepackage[cyr]{aeguill}
\usepackage{geometry}
\geometry{verbose,letterpaper,tmargin=27.5mm,bmargin=27.5mm,lmargin=27.5mm,rmargin=27.5mm}
\usepackage{graphicx}
\usepackage{epigraph}
\setlength\epigraphwidth{13cm}
\usepackage[center,up,labelfont=bf]{caption}
\usepackage{float}
\usepackage{url}
\newcommand{\guil}[1]{?~{#1}~?}    %guillemets 
\newcommand{\guill}[1]{``{#1}''}     %guillements dans les guillemets
\bibliography{bibliographie/bibliographiebdd}

\pagenumbering{arabic}
\setcounter{page}{1}


\begin{document}



\sloppy
\begin{titlepage}
  \begin{singlespace}
\begin{center}
{Université de Lille} \vspace{1.5 cm}\\
\end{center}

\begin{center}


\Large{{\bf{Modèle de Potts pour la classification}}\\Master Mathématiques et Applications}


\end{center}
\vspace{1.5 cm}
\begin{center}
\normalsize{par Khalifa Naïl et Briouat Farid}
\vspace{1.5 cm}
\end{center}

\begin{center}
Département de Mathématiques\\
Faculté des sciences et technologies
\end{center}
\vspace{1.5 cm}

\begin{center}
  Mémoire présenté à la Faculté des sciences et technologies en vue de l'UE TER
\end{center}
\vspace{1.5 cm}






\begin{center}
Mai 2023\\
\vspace{3 cm}
KHALIFA Naïl et BRIOUAT Farid
\end{center}
  \end{singlespace}

  \newpage
\end{titlepage} 
%\input{titres/garde.tex}


\addcontentsline{toc}{chapter}{Remerciements}
\chapter*{Remerciements} 

Nous remercions Monsieur Nicolas Wicker pour son encadrement et son aide précieuse.
Nous remercions Madame Charlotte Baey notre enseignante de statistique computationnelle.



\begin{singlespace}
\tableofcontents % table des mati?res
\listoffigures % si vous avez des images... si vous n'en avez pas, effacez cette ligne
\end{singlespace}



%dans les fichier vous trouverez des exemples d'usage des diff?rentes commandes de LaTeX

%\addcontentsline{toc}{chapter}{Introduction}
\chapter*{Introduction}

\begin{document}
Etant étudiants en master Mathématiques et Application dans la section Ingénierie Statistique Numérique, l'étude d'évolution de cluster et de simulation grâce à des algorithmes MCMC\footnote{Markov chain Monte Carlo} nous semble assez logique.
Le but de notre TER était de réaliser une classification non supervisée d'un ensemble de point en utilisant le modèle de Potts.
On a donc procédé en deux étapes une premiere étape qui était l'implémentation d'un programme informatique en C++ et en Python.
Ces deux langages sont très utiles pour la modélisation qui se fera en Python et pour l'algorithmique qui se fera en C++.
Puis une deuxième partie qui est l'étude de l'algorithme de Metropolis-Hasting, sa convergence et l'utilité de cet algorithme dans un tel modèle.

\end{document} %intro

%%! Author = nail
%! Date = 09/03/2023

% Preamble
\documentclass[11pt]{article}

\addcontentsline{toc}{chapter}{Programmation}
\chapter*{Programmation}
\setcounter{chapter}{2}
\newcommand\tab[1][0.8cm]{\hspace*{#1}}

% Document
\begin{article}

test

\end{article} %explication de la partie prog

%%! Author = nail
%! Date = 09/03/2023

% Preamble
\documentclass[11pt]{article}

% Packages
\usepackage{amsmath}

% Document
\begin{document}



\end{document} %explication algorithme metropolis hasting


%\printbibheading %exemple de bibliographie divis?e en sections. Pour ajouter des oeuvres non cit?es,utiliser \nocite

%\printbibliography[keyword=pratique,heading=subbibliography,title={Th?ories litt?raires dans les jeux vid?o}]
%\printbibliography[keyword=litteraire,heading=subbibliography,title={Narratologie et structuralisme}]

%\printbibliography[keyword=jeu,heading=subbibliography,title={\emph{Games studies}}]


\end{document} 